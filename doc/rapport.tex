\documentclass[a4paper,12pt]{article} \usepackage{graphicx}
%\usepackage{epstopdf} %\usepackage{gensymb} \usepackage{longtable}
\usepackage{graphicx}
\usepackage{listings}
\usepackage{caption}
\usepackage{subcaption}
\usepackage[english,swedish]{babel}
\usepackage[utf8]{inputenc}
\usepackage[T1]{fontenc}

\selectlanguage{swedish}
\title{Interactive Solar System}
\author{Gustav Svensk, Johan Jönsson, \\
    Nora Björklund, Christopher Hallberg}
\date{\today}

\begin{document}
\maketitle
\selectlanguage{english}
\tableofcontents
\newpage
\section{Introduction}
% Lista will do och might do

\section{How To Use}
\subsection{Setup}
% Hur man kompilerar allt, med soil och allt sånt.
\subsection{Usage}
% Hur man startar programmet, inställningar och hur man styr

\section{Implementation}
\subsection{Overview}
% UML karta
\subsection{Celestial Bodies}
% Kanske mer i ordets rätta bemärkelse snarare än structen Cel_bodies?. dvs asteroider.
% Behövs troligtvis inte subsubsections
\subsection{Spacebox \and Ship (Cat?)}
\subsection{Physics}
\subsection{Collision}
\subsection{Camera \and Frustum}

\section{Problems}
\subsection{Frustum Culling \and Up-vector}
\subsection{Multiple Suns}

\section{Conclusions}

\end{document} 
% Local Variables: %%% mode: latex %%% TeX-master: t %%% End:
