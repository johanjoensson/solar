\documentclass[a4paper,12pt]{article} \usepackage{graphicx}
\usepackage{graphicx}
\usepackage{listings}
\usepackage{caption}
\usepackage{subcaption}
\usepackage[english,swedish]{babel}
\usepackage[utf8]{inputenc}
\usepackage[T1]{fontenc}
\lstset{language=c++,
        numberstyle=\footnotesize,
        basicstyle=\ttfamily\footnotesize,
        breaklines=true
    }


\selectlanguage{swedish}
\title{Interactive Solar System}
\author{Gustav Svensk, Johan Jönsson, \\
    Nora Björklund, Christopher Hallberg}
\date{\today}

\begin{document}
\maketitle
\selectlanguage{english}
\tableofcontents
\newpage
\section{Introduction}
% Lista will do och might do
Our finished program is an interactive solar system that follows a subset of 
the laws of physics. The player can move around in the system and configure
initial state of the system.

\subsection{Will Do}
\begin{itemize}
        \item Bodies with physical properties
        \item Newtonian physics
        \item Configurable initial state
        \item Collision handling
        \item Movable camera
        \item Light source(s)
        \item Spacebox (skybox but in SPACE!)
        \item Asteroid belt
        \item Controllable spaceship
\end{itemize}

\subsection{Might Do}
\begin{itemize}
        \item Random initial state
        \item Relativistic physics (special)
        \item Drawing optimized for frustum
        \item No sound (because SPACE)
        \item Landing on selected bodies (sound?)
        \item Build our solar system
        \item Trans-neptunian objects (Pluto?, Death star?)
        \item Aliens
\end{itemize}

\section{How To Use}
This section describes how to install and run the program.
\subsection{Setup}
% Hur man kompilerar allt, med soil och allt sånt.

\subsection{Usage}
The settings for the initial state can be seen in the table below.

        \begin{tabular}{| l | p{9cm} |}
                \hline
                \textbf{Flag} & \textbf{Action} \\
                \hline
                -h & Display a help message \\
                -s & Creates a sun \\
                -p $np$ & Creates $np$ planets \\
                -r $radius$ & Set the maximum distance from the origin in which planets can be created to $radius$\\
                -m $mass$ & Set the maximum mass of the planets to $mass$\\
                -n $mass$ & Set the maximum mass of the sun to $mass + 1E10$ \\
                -v $vel$ & Set the maximum initial velocity of the planets to $vel$ \\
                -a $na$ & Create an asteroid belt with $na$ asteroids \\
                \hline
        \end{tabular}
\end{center}
% Hur man startar programmet, inställningar och hur man styr

In the game the ``wasd'' buttons and the mouse are used for navigation.
Other keys can be seen in the table below.
\begin{center}
        \begin{tabular}{|c|l|}
                \hline
                \textbf{Key} & \textbf{Action} \\
                \hline
                p & Save a screenshot to ``space.bmp'' \\
                g & Toggle the locking of keyboard and mouse \\
                left arrow & Decrease simulation step (slower, but more accurate simulation) \\
                right arrow & Increase simulation step (faster, but less accurate simulation) \\
                up arrow & Increase ship speed \\
                down arrow & Decrease ship speed \\
                \hline
        \end{tabular}
\end{center}

\section{Implementation}
\subsection{Overview}
% UML karta
% Vilket språk?
% Vilka bibliotek?
\subsection{Celestial Bodies}
% Kanske mer i ordets rätta bemärkelse snarare än structen Cel_bodies?. dvs asteroider.
% Behövs troligtvis inte subsubsections
The celestial bodies are the objects that are affected by gravity in
our system i.e the planets and the sun. They are all subclasses of the
class object. The planets are of the class body which inherits from
object. Body's attributes and functions can be seen in figure
\ref{fig:XX}. The sun is very similar to the planets except that it
also emits sun light. The sun therefore is a subclass that inherits from
body \ref{fig:YY}. 

\subsection{Spacebox \and Ship (Cat?)}
The drawing of the spacebox and the ship is handled specially in the shaders.
This is because the spacebox is always drawn around the camera and is therefore
only dependent of the rotation part of the camera matrix and the lighting does not
affect the spacebox. The ship is always drawn in front of the camera and is not dependent on the camera matrix.
To let the shaders know that the object drawn is a spacebox or a ship a uniform
int is sent to the shaders. Examples can be seen in listings \ref{lst:ship}
and \ref{lst:vert}.
\begin{lstlisting}[caption={Draw spaceship}, label={lst:ship}]
glUniform1i(glGetUniformLocation(program, "spaceship"), 1);
...
Draw ship here
...
glUniform1i(glGetUniformLocation(program, "spaceship"), 0);
\end{lstlisting}

\begin{lstlisting}[caption={Vertex shader}, label={lst:vert}]
if(spacebox == 0 && spaceship == 0){
    gl_Position = proj_matrix * cam_matrix * mdl_matrix * vec4(in_position, 1.0);
} else if(spaceship == 1 && spacebox == 0)
    gl_Position = proj_matrix * mdl_matrix * vec4(in_position, 1.0);
else {
    gl_Position = proj_matrix * mat4(mat3(cam_matrix)) * mdl_matrix * vec4(in_position, 1.0);
} 
\end{lstlisting}

\subsection{Physics}
\subsection{Collision}
\subsection{Camera \and Frustum}

\section{Problems}
\subsection{Frustum Culling \and Up-vector}
\subsection{Multiple Suns}
It is possble to implement two suns at the moment but the light from
them will not be correct. This happens because we upload the light
from the celestial bodies list, and in the fragment shader the light
will be calculated according to the latest uploaded light source only and not both of them. To
solve this we need to upload the light as an array and then calculate
the light combined. For the array solution to work the number of suns
needs to be limited too since it is not possible to have a array of
undecided length in the fragment shader.
\section{Conclusions}

\end{document} 
% Local Variables: %%% mode: latex %%% TeX-master: t %%% End:
